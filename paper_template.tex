\documentclass[conference]{IEEEtran}
\usepackage{times}

% numbers option provides compact numerical references in the text. 
\usepackage[numbers]{natbib}
\usepackage{multicol}
\usepackage[bookmarks=true]{hyperref}

% \pdfinfo{
%   /Author (Team 24)
%   /Title  (EcoSense: Smart CO2 Monitoring for Sustainable Environments)
%   /CreationDate (D:202401170000)
%   /Subject (IoT)
%   /Keywords (IoT;Protocols;C02)
% }

\begin{document}

% paper title
\title{EcoSense: Smart CO2 Monitoring for Sustainable Environments}

\author{
        Woon Jun Wei \textit{2200624} \\
        Benjamin Loh Choon How \textit{2201590} \\
        Wang Rongqi Richie \textit{2201942} \\
        Poon Xiang Yuan \textit{2200559} \\
        Low Hong Sheng Jovian \textit{2203654}\\
    }
    
%\author{\authorblockN{Michael Shell}
%\authorblockA{School of Electrical and\\Computer Engineering\\
%Georgia Institute of Technology\\
%Atlanta, Georgia 30332--0250\\
%Email: mshell@ece.gatech.edu}
%\and
%\authorblockN{Homer Simpson}
%\authorblockA{Twentieth Century Fox\\
%Springfield, USA\\
%Email: homer@thesimpsons.com}
%\and
%\authorblockN{James Kirk\\ and Montgomery Scott}
%\authorblockA{Starfleet Academy\\
%San Francisco, California 96678-2391\\
%Telephone: (800) 555--1212\\
%Fax: (888) 555--1212}}


% avoiding spaces at the end of the author lines is not a problem with
% conference papers because we don't use \thanks or \IEEEmembership


% for over three affiliations, or if they all won't fit within the width
% of the page, use this alternative format:
% 
%\author{\authorblockN{Michael Shell\authorrefmark{1},
%Homer Simpson\authorrefmark{2},
%James Kirk\authorrefmark{3}, 
%Montgomery Scott\authorrefmark{3} and
%Eldon Tyrell\authorrefmark{4}}
%\authorblockA{\authorrefmark{1}School of Electrical and Computer Engineering\\
%Georgia Institute of Technology,
%Atlanta, Georgia 30332--0250\\ Email: mshell@ece.gatech.edu}
%\authorblockA{\authorrefmark{2}Twentieth Century Fox, Springfield, USA\\
%Email: homer@thesimpsons.com}
%\authorblockA{\authorrefmark{3}Starfleet Academy, San Francisco, California 96678-2391\\
%Telephone: (800) 555--1212, Fax: (888) 555--1212}
%\authorblockA{\authorrefmark{4}Tyrell Inc., 123 Replicant Street, Los Angeles, California 90210--4321}}


\maketitle

\begin{abstract}
"EcoSense" is an innovative Internet of Things project that presents a smart and sustainable method to CO2 monitoring, with the goal of contributing to environmental well-being. Using inexpensive microcontrollers and innovative sensors, the system analyzes CO2 levels in real time, emphasizing scalability and efficiency. By combining the LoRa, WiFi or Ethernet protocols, data transmission is guaranteed to be smooth, energy consumption is optimized, and extensive deployment is made possible. In addition to ensuring prompt interventions when predetermined thresholds are surpassed, an alarm system facilitates easy visualization and interpretation of CO2 levels by stakeholders. "EcoSense" has the potential to promote sustainable habits and offer significant insights into mitigating global warming.
\end{abstract}

\IEEEpeerreviewmaketitle

\section{Introduction}

\subsection{Problem Statement}
This project endeavors to tackle the imperative requirement for an efficient and scalable CO\textsubscript{2} monitoring solution in environmental domains. Leveraging IoT technologies, the objective is to enable real-time analysis and offer insightful data for the sustainable management of atmospheric CO\textsubscript{2} levels. The scalability aspect facilitates the deployment of distributed nodes nationwide, empowering countries to comprehensively assess air quality, draw informed conclusions, and make strategic decisions in the collective fight against Global Warming.

\section{Project Specifications}

\subsection{Project Architecture}

\subsection{Project Materials}

\begin{table}[htbp]
  \caption{Materials}\label{tab:materials}
  \centering
  \begin{tabular}{p{4.5cm}|p{4.5cm}}
    \hline
    \textbf{Controllers (MCUs)} & \textbf{Sensors} \\
    \hline
    Raspberry Pi 4B & MQ-135 Air Quality Sensor \\
    \hline
    \begin{tabular}[t]{@{}l@{}}
      Raspberry Pi Pico \textbf{OR}\\
      M5 Stick C Plus \textbf{OR}\\
      Arduino Uno
    \end{tabular}
    &
    LORA HAT(s) \\
    \hline
    & DHT11 Sensor and GPS Sensor (\textbf{Optional}) \\
    \hline
  \end{tabular}
\end{table}

These are a list of project materials that we will be using for our project. 

\subsection{Planned Protocols}

\begin{table}[htbp]
  \caption{Protocols}\label{tab:protocols}
  \centering
    \begin{tabular}[c]{l|l}
      \hline
      \multicolumn{1}{c|}{\textbf{Protocol}} & 
      \multicolumn{1}{c}{\textbf{Use}} \\
      \hline
      Wifi/Ethernet & Communication between Raspberry Pi and Server \\
      \hline
      LoRa/BLE & Communication between Raspberry Pi and Microcontrollers \\
      \hline
    \end{tabular}
\end{table}


\subsection{Functional Requirements}
Lorem ipsum dolor sit amet, officia excepteur ex fugiat reprehenderit enim labore culpa sint ad nisi Lorem pariatur mollit ex esse exercitation amet. Nisi anim cupidatat excepteur officia. Reprehenderit nostrud nostrud ipsum Lorem est aliquip amet voluptate voluptate dolor minim nulla est proident. Nostrud officia pariatur ut officia. Sit irure elit esse ea nulla sunt ex occaecat reprehenderit commodo officia dolor Lorem duis laboris cupidatat officia voluptate. Culpa proident adipisicing id nulla nisi laboris ex in Lorem sunt duis officia eiusmod. Aliqua reprehenderit commodo ex non excepteur duis sunt velit enim. Voluptate laboris sint cupidatat ullamco ut ea consectetur et est culpa et culpa duis.



% \section{Section}

% Section text here. 

% \subsection{Subsection Heading Here}
% Subsection text here.

% \subsubsection{Subsubsection Heading Here}
% Subsubsection text here.


% \section{RSS citations}

% Please make sure to include \verb!natbib.sty! and to use the
% \verb!plainnat.bst! bibliography style. \verb!natbib! provides additional
% citation commands, most usefully \verb!\citet!. For example, rather than the
% awkward construction 

% {\small
% \begin{verbatim}
% \cite{kalman1960new} demonstrated...
% \end{verbatim}
% }

% \noindent
% rendered as ``\cite{kalman1960new} demonstrated...,''
% or the
% inconvenient 

% {\small
% \begin{verbatim}
% Kalman \cite{kalman1960new} 
% demonstrated...
% \end{verbatim}
% }

% \noindent
% rendered as 
% ``Kalman \cite{kalman1960new} demonstrated...'', 
% one can
% write 

% {\small
% \begin{verbatim}
% \citet{kalman1960new} demonstrated... 
% \end{verbatim}
% }
% \noindent
% which renders as ``\citet{kalman1960new} demonstrated...'' and is 
% both easy to write and much easier to read.
  
% \subsection{RSS Hyperlinks}

% This year, we would like to use the ability of PDF viewers to interpret
% hyperlinks, specifically to allow each reference in the bibliography to be a
% link to an online version of the reference. 
% As an example, if you were to cite ``Passive Dynamic Walking''
% \cite{McGeer01041990}, the entry in the bibtex would read:

% {\small
% \begin{verbatim}
% @article{McGeer01041990,
%   author = {McGeer, Tad}, 
%   title = {\href{http://ijr.sagepub.com/content/9/2/62.abstract}{Passive Dynamic Walking}}, 
%   volume = {9}, 
%   number = {2}, 
%   pages = {62-82}, 
%   year = {1990}, 
%   doi = {10.1177/027836499000900206}, 
%   URL = {http://ijr.sagepub.com/content/9/2/62.abstract}, 
%   eprint = {http://ijr.sagepub.com/content/9/2/62.full.pdf+html}, 
%   journal = {The International Journal of Robotics Research}
% }
% \end{verbatim}
% }
% \noindent
% and the entry in the compiled PDF would look like:

% \def\tmplabel#1{[#1]}

% \begin{enumerate}
% \item[\tmplabel{1}] Tad McGeer. \href{http://ijr.sagepub.com/content/9/2/62.abstract}{Passive Dynamic
% Walking}. {\em The International Journal of Robotics Research}, 9(2):62--82,
% 1990.
% \end{enumerate}
% %
% where the title of the article is a link that takes you to the article on IJRR's website. 


% Linking cited articles will not always be possible, especially for
% older articles. There are also often several versions of papers
% online: authors are free to decide what to use as the link destination
% yet we strongly encourage to link to archival or publisher sites
% (such as IEEE Xplore or Sage Journals).  We encourage all authors to use this feature to
% the extent possible.

% \section{Conclusion} 
% \label{sec:conclusion}

% The conclusion goes here.

% \section*{Acknowledgments}

%% Use plainnat to work nicely with natbib. 

% \bibliographystyle{plainnat}
% \bibliography{references}

\end{document}


